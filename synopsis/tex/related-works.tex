\sectionLabel*{Related Work}

Methods for physics-based character animation that use forward dynamic
simulations have been a research focus for over two decades, most often with
human locomotion as the motion of interest. A survey of the work in this area
can be found in \cite{geijtenbeek2012interactive}. Impressive results were
achieved in classical works by \cite{coros2011locomotion},
\cite{wang2012optimizing} and \cite{geijtenbeek2013flexible}. But most of these
works focus on controlling physics-based locomotion over flat terrain. Recently
there have been some very interesting attempts to use deep reinforcement
learning for locomotion on complex terrain, for example \cite{peng2016terrain},
but only in 2D at the moment. Another interesting approach using biologically
inspired idea that motor systems are hierarchical was suggested in
\cite{geijtenbeek2012interactive}. Animation researchers have been interested
in the control of locomotion for 3D humanoid characters for almost 20 years.
One important recent contribution is \emph{SIMBICON} \cite{yin2007simbicon}, a
remarkably robust 3D humanoid locomotion controller based on the balance
control of \cite{raibert1991animation}. A number of projects have since focused
on expanding the controller repertoire for simulated bipeds and on locomotion
in complex environments. At the same time, efforts have been made to make the
synthesized motions more human-like, or ``natural''. The original
\emph{SIMBICON}-style controllers tend to produce gaits lacking hip extension
with a constant foot orientation. Knee angles lack flexion during swing, but
lack extension at heel strike. More recent controllers improve motions by
designing better target trajectories in joint or feature space. While more
human-like ankle motions have been produced, differences in the hip and knee
angles persist. Perhaps more importantly, controllers relying on hand-tuned
trajectories cannot be easily used to investigate how the control strategies
change with respect to new constraints. For example, how would the character's
motion style change given a physical disability? Can we synthesize appropriate
gaits for older or younger characters? Impressive results have also been
achieved by controllers based on tracking motion capture data as in
\cite{da2008interactive} and \cite{ye2010optimal}. However, as with methods
that tune joint trajectories or controller parameters by hand, motion capture
driven controllers have a limited ability to predict changes in gait.
